\documentclass{article}

\usepackage[final]{nips_2017}

\usepackage[utf8]{inputenc} % allow utf-8 input
\usepackage[T1]{fontenc}    % use 8-bit T1 fonts
\usepackage{hyperref}       % hyperlinks
\usepackage{url}            % simple URL typesetting
\usepackage{booktabs}       % professional-quality tables
\usepackage{amsfonts}       % blackboard math symbols
\usepackage{nicefrac}       % compact symbols for 1/2, etc.
\usepackage{microtype}      % microtypography
\usepackage[ruled,linesnumbered]{algorithm2e} % algorithms
\usepackage[]{amsmath}
\usepackage{graphicx}
\usepackage[english]{babel}
\usepackage{color}



\usepackage{blindtext}
\newcommand{\indep}{\rotatebox[origin=c]{90}{$\models$}}
\newtheorem{theorem}{Theorem}[section]
\newtheorem{corollary}{Corollary}[theorem]
\newtheorem{lemma}[theorem]{Lemma}
\newtheorem{definition}[theorem]{Definition}
\newenvironment{proof}{\paragraph{Proof:}}{\hfill$\square$}

\title{%Midterm Update:\\
Offline Heterogeneous Policy Evaluation for Contextual Bandits: A Causal Approach}

\author{
  Amanda Coston \quad Liu Leqi\\
  Machine Learning Department\\
  Carnegie Mellon University\\
  Pittsburgh, PA 15213 \\
  \texttt{\{acoston, leqil\}@cs.cmu.edu} \\
}
\begin{document}
\maketitle
\begin{abstract}
    Offline policy evaluation is integral to understanding how a policy will perform before investing time and money in online evaluation. Whereas previous work has investigated methods for computing how well a policy does on average (i.e. value function) in an offline evaluation, this paper goes further by proposing and analyzing methods for offline policy evaluation on a more granular, subgroup-specific scale. We propose a doubly robust estimator of $\mathcal{S}$-specific value function, where $\mathcal{S}$ defines a subset of our covariate space $\mathcal{X}$. Our estimator is robust to parametric mispecification in either of the nuisance functions. In a nonparametric setting, our estimator is $\sqrt{n}$-consistent and asymptotically normal under assumptions of sample-splitting and $n^{1/4}$ convergence in the error terms of the nuisance functions. This estimator can be used to compute on the $\mathcal{S}$-specific value function of a subgroup that is of a priori interest (such as a vulnerable or legally protected population); alternatively, we propose a tree-based method to empirically find the subgroups that are most affected under the proposed policy. We show how rejection sampling can be used with our doubly robust estimator to evaluate nonstationary (or online) policies. Evaluating $\mathcal{S}$-specific value functions is useful in assessing how fair a policy is in any setting where we have a randomized policy assignment conditional on covariates, including clinical trials as well as traditional contextual bandit applications such as online advertising and content recommendations.
\end{abstract}
\section{Introduction}

In conditionally randomized experiments, such as clinical trials, researchers often have to specify the probability of assigning treatments based on measured features/covariates. In order for researchers to determine how to assign treatments--which following the contextual bandit literature, we will denote as a \textit{policy}--they need to understand how well the policy will perform in terms of a key outcome of interest, such as patient outcomes in a clinical trial. Note that in bandit/reinforcement contexts, this outcome is called a \textit{reward}. Questions of interest include how the policy will perform on average, how the policy will affected certain "protected" or vulnerable groups (such as children), and who will benefit most and who will suffer most under a given policy. A policy can be stationary, where stationary denotes that the policy is constant over time; or a policy can be nonstationary, meaning that it updates its assignment mechanism based on the observed history. Our paper considers both types of policies. \\

 \cite{dudik2014doubly} propose a doubly robust estimator for offline policy evaluation that given a history of observations under an old policy, estimates the average expected reward under a new "target" policy. The average expected reward under a policy is known as the value function. Their estimator is unbiased for the value function even if one of the two nuisance functions is biased (or misspecified)--hence the "double robustness." Doubly robust estimators are often regarded as the optimal estimators in causal inference because of their robustness to bias and because their rate of convergence depends on a product of errors of the nuisance terms. Thus, they can achieve fast rates of convergence under milder requirements on the convergence rates of the nuisance functions. \\

Despite the advantages of the doubly robust estimator, there are limitations with \cite{dudik2014doubly}'s method of offline policy evaluation. Coarse summaries like an average obfuscate many details about the policy's performance. They do not tell us where/why the policy is performing well (or poorly). They cannot help us analyze how fair or equitable a given policy is. Consider as a trivial example comparing two policies that have the same average outcome $\Psi$ where under policy $A$ all patients are treated with a "good" treatment and all patients have outcome $\Psi$ versus under policy $B$, half the patients are assigned an "excellent" treatment and therefore have outcome $2\Psi$ but the other half are assigned no treatment and have outcome = 0. While the average may be the same, the fairness of the two policies are drastically different. \\

Recent advances in causal inference have moved beyond estimating the average treatment effect (ATE) to estimating conditional or heterogeneous treatment effects (HTE), which considers the treatment effect for a particular subgroup as described by a set of covariates. These methods tell us the covariates (equivalently, contexts in the bandit setting) where the treatment works well and where the treatment does not work or even performs poorly. Applying insights from these methods can help us better understand our target policy's predicted performance. Recent papers focus on estimating effects of continuous random variables in the binary treatment setting. \cite{wager2017estimation} and \cite{athey2016recursive} use decision trees and random forests to estimate HTE's. They prove that their estimator is consistent and asymptotically normal under assumptions that may be hard to meet in practice, such as scaling the subsampling rate appropriately with the sample size. \cite{imai2013estimating} use support vector machines to estimate treatment effect heterogeneity in a randomized binary treatment trial. They require parametric assumptions and do not analyze the statistical properties of their estimator. \\

In this paper we propose and analyze a doubly robust estimator for subgroup policy evaluation in an offline setting, which we term $\mathcal{S}$-specific value functions. Our estimator is unbiased as long as one of two nuisance functions is estimated consistently. We show that our estimator can achieve $\sqrt{n}$-consistency and asymptotic normality under assumptions of positivity, consistency, exchangeability, sample-splitting, smoothness or sparsity in the nuisance functions, and a requirement that the estimates of the old policy be bounded away from zero. In certain applications there may be clear subgroups that are a priori of interest, such as protected groups or minorities; whereas in other settings we may be interested in empirically discovering which subgroups are most or least affected by the policy. For the latter case, we propose a tree-based method for finding subgroups of interest based on our doubly robust estimates of the $\mathcal{S}$-specific value functions.  \\

The paper is organized as follows: In section \ref{bandits} we review the literature on doubly robust offline policy evaluation. In section \ref{drest} we propose and analyze our $\mathcal{S}$-specific doubly robust estimator in the stationary setting. In section \ref{finding}, we propose a tree-based method for finding subgroups that benefit most and that suffer most under a new policy. Finally, in section \ref{nonstationary}, we show how our doubly robust $\mathcal{S}$-specific estimator can be used to evaluate nonstationary policies by modifying Algorithm 1 in \cite{dudik2014doubly}. \\

Offline policy evaluation is important in applications ranging from clinical trials and personalized medicine to social policy-making and online advertising. Offline heterogeneous policy evaluation enables us to gain deeper insight into the policy's performance and to identify the subgroups most at risk under a policy, which is essential in assessing a policy's fairness.

% Let's save these definitions for the section below so that we can define things like action, etc properly? But feel free to add to the medical example above to make the connection clear. "the patient’s current characteristics are contextual infor- mation, a treatment is an action, and a DTR is a policy. Similar to contextual bandits, the quantity of interest in DTR can be expressed by a numeric reward signal related to the clinical outcome of a treatment." 
%Hence, we would like to apply heterogeneous treatment effect estimation regime to contextual bandits. Such a heterogeneous policy evaluation is important because...


%$
%&= \frac{1}{\sum_{k=1}^N \mathbb{I}{\{x_k \in \mathcal{S} \}}}  \sum_{k=1}^{N } \mathbb{I}{\{x_k \in \mathcal{S} \}} \phi(\pi, x_k)  - \mathbf{P}_{X_\mathcal{S}} \left[ \phi(\pi, x_k) \right] \label{eq1}\\
%&+ \mathbf{P}_{X_\mathcal{S}} \left[ \widehat{\phi}(\pi, x_k) \right] - \mathbf{P}_{X_\mathcal{S}} \left[ \phi(\pi, x_k) \right]  \label{eq2}\\
%&+ \left( \frac{1}{\sum_{k=1}^N \mathbb{I}{\{x_k \in \mathcal{S} \}}}  \sum_{k=1}^{N} \mathbb{I}{\{x_k \in \mathcal{S} \}} \left[ \widehat{\phi} (\pi, x_k) - \phi(\pi, x_k) \right] - \mathbf{P}_{X_\mathcal{S}} \left[ \widehat{\phi}(\pi, x_k) - \phi(\pi, x_k)  \right] \right). \label{eq3}$

 
\section{Contextual Bandit Approaches for Offline Policy Evaluation} \label{bandits}
%add definitions of nonstationary and stationary policies
Contextual bandits algorithms specify policies in settings where we need to make sequential decisions and where rewards are a function of observed contexts and actions taken by the decision-maker. Formally, we consider the below setting: 
At time $k$, 
\begin{itemize}
    \item A context is specified by a vector $x_k \in \mathcal{X}.$
    \item An action $a_k$ is chosen from $\mathcal{A}$ based on $x_k$ and the past history $\{z_i\}_{i=1}^{k-1}$.
    \item A reward $r_k \in [0,1]$ for taking action $a_k$ in context $x_k$ is then assigned. 
    \item A new history record $z_k = (x_k, a_k, r_k)$ is formed.
\end{itemize}
% do we have the positivity assumption somewhere 
Traditional assumptions in this setting include \cite{dudik2014doubly}: i) Contexts $x_1, ..., x_N$ are drawn i.i.d. from a distribution; ii) $|\mathcal{A}|$ is finite; iii) the distribution of the reward $R|A, X$ is unknown but does not change over time. A policy is a function, defined as $\pi: \mathcal{X} \times \mathcal{A} \to [0,1]$ where $\pi(a|x)$ is the probability of assigning action $a$ to context $x$. In the clinical example, the policy is how doctors assign patients treatment; the decision-maker is the doctor; the action is the treatment; the context is the patient's features including demographic and health information; the reward is patient outcomes (such as recovery from disease).~\cite{dudik2014doubly} assumes that there is an underlying policy $\pi_z$ that the agent follows when producing the history.

With this,~\cite{dudik2014doubly} defines the goal of offline policy evaluation. Given the history $\{z_i\}_{i=1}^N$ and a new policy $\pi$, we estimate the value of the new given policy on the set of encountered contexts:
$$ V(\pi, \mathcal{X}) = \mathbf{E}_{X}\mathbf{E}_{A}\mathbf{E}_{R}[R].$$

Note that the expectation is taken over $X$, which means that the value of a policy is evaluated on average over the distribution of contexts. That is, $V(\pi, \mathcal{X})$ is evaluating the average treatment effect (ATE) of the given policy $\pi$, from which we sample an action given a context. %Here, we provide a brief summary on the doubly robust estimator people used in estimating ATE in contextual bandit. 

There are two standard methods in estimating $V(\pi, \mathcal{X})$, namely the direct method (DM) and inverse propensity score (IPS)~\cite{dudik2014doubly}. The direct method uses below estimator:
$$\widehat{V}_{DM}(\pi, \mathcal{X}) = \frac{1}{N}{\sum_{k=1}^N}\sum_{a \in \mathcal{A}} \pi(a|x_k)\hat{r}(x_k, a),$$ where $\hat{r}(x_k, a)$ is an estimator of $\mathbf{E}_{R}[R|X,A]$ which can be estimated using standard nonparametric regression methods. A drawback to DM is that since $\hat{r}$ has no information on $\pi$, it might only estimate well at places that are not important to $\pi$.The second approach defines inverse propensity score $\frac{\pi(a|x)}{\pi_{z}(a|x)}$ where $\pi$ is the new given policy and $\pi_{z}$ is the history collecting policy. The estimator is then defined as
$$\widehat{V}_{IPS} (\pi, \mathcal{X}) = \frac{1}{N}\sum_{k=1}^N \sum_{a \in \mathcal{A}}\mathbb{I}\{A_k = a\} \frac{\pi(a|x_k)}{\hat{\pi}_{z}(a|x_k)} \cdot r(x_k, a),$$ where $\hat{\pi}_{z}(a|x_k)$ is an estimate of $\pi_{z}(a|x_k)$ and $r(x_k, a_k)=r_k$. The doubly robust estimator in \cite{dudik2014doubly} combines the two:
$$\widehat{V}_{DR} (\pi, \mathcal{X}) = \frac{1}{N}\sum_{k=1}^N \left[ \sum_{a \in \mathcal{A}} \pi(a|x_k)\hat{r}(x_k, a) + \sum_{a \in \mathcal{A}} \mathbb{I}_{\{A_k = a\}}\frac{\pi(a|x_k)}{\hat{\pi}_{z}(a|x_k)} \cdot (r(x_k, a) - \hat{r}(x_k, a))\right].$$



%\cite{dudik2014doubly} established finite-sample error bound for $\widehat{V}_{DM}$. With high probability, and under some standard assumptions, $|\widehat{V}_{DR} - V| \leq O(\frac{1}{n}).$ (Need to update)


$\widehat{V}_{DR} (\pi, \mathcal{X})$ is unbiased if either $\hat{r}(x_k, a)$ or $\hat{\pi}_{z}(a|x_k)$ is unbiased. However, we note a major limitation to the average treatment effect $V_{DR} (\pi, \mathcal{X})$: it does not tell us in which context $\pi$ does well and where it might do poorly. To do so, we would like to estimate the heterogeneous treatment effect of the policy in different subset $\mathcal{S}$ of the contexts, i.e. $V_{DR} (\pi, \mathcal{S})= \mathbf{E}_{X_\mathcal{S}} \mathbf{E}_{A}\mathbf{E}_{R} [R].$ This estimator is valid for any subset $\mathcal{S}$ such that the size of $\mathcal{S}$ goes to infinity as $N$ goes to infinity.
 
 %At the extreme, $\mathcal{S}$ can consist only a single context and  $V_{DR} (\pi, \mathcal{S})$ will be the treatment effect under that specific context.

%To our best knowledge, such an offline heterogeneous policy evaluation in contextual bandit has not been done yet; however, such methods are of recent interest in causal inference literature.


%\input{sections/hte.tex}
\section{Doubly Robust Estimator for $\mathcal{S}$-specific Value Functions} \label{drest}
%\textcolor{red}{Mention assumptions here}

In this analysis for simplicity we assume that we are evaluating a new stationary policy. In section \ref{nonstationary} we extend this to the nonstationary policy setting. Let $\pi$ denote our new target policy and $\pi_z$ denote the historical policy for which we have observations. We want to estimate the value function $V(\pi, \mathcal{S})$ of a new policy $\pi$ on contexts $x$ that are in a subset $\mathcal{S}$ of $\mathcal{X}$. Formally, we want to estimate
\begin{eqnarray*}
 V(\pi, \mathcal{S}) &=& \mathbf{E}_{X_\mathcal{S}} \mathbf{E}_{A}\mathbf{E}_{R} [R]\\
 &=& \mathbf{E}_{X_\mathcal{S}} \left[ \sum_{a \in \mathcal{A}} \pi(a|x) \mathbf{E}_{R}[R|X_\mathcal{S} = x,A=a] \right]\\
 &=& \mathbf{E}_{X_\mathcal{S}} \left[ \sum_{a \in \mathcal{A}} \pi(a|x) r(x, a)  \right].
\end{eqnarray*}

where $X_{\mathcal{S}}$ denotes a context in $\mathcal{S}$ (for example, if $\mathcal{S}$ denotes females, then $X_{\mathcal{S}}$ must be a woman), $A$ denotes the action chosen according to the context $X_{\mathcal{S}}$ under policy $\pi$ and $R$ is the obtained reward.  %make sure we've defined other terms like A in previous section

The proposed estimator is 
\begin{align*} 
\widehat{V}_{DR}(\pi, \mathcal{S}) = &\frac{1}{\sum_{k=1}^N \mathbb{I}{\{x_k \in \mathcal{S} \}}}  \sum_{k=1}^{N } \mathbb{I}{\{x_k \in \mathcal{S} \}} \\
&\times \left[\sum_{a \in \mathcal{A}} \pi(a|x_k)\hat{r}(x_k, a) + \sum_{a \in \mathcal{A}} \mathbb{I}_{\{A_k = a\}}\frac{\pi(a|x_k)}{\hat{\pi}_{z}(a|x_k)} \cdot (r(x_k, a) - \hat{r}(x_k, a)) \right]
 \end{align*}%.$$%\widehat{V}_{DR}
where ${\{x_k, a_k, r(x_k, a_k)\}}_{k=1}^N$ are the history contexts, actions and rewards such that $r(x_k, a_k) = r_k$. The random variable $A_k$ denotes the action assigned in history at time $k$. $\hat{\pi}_z(a|x_k)$ is the estimator of $\mathbf{E}[\mathbb{P}(A_k=a|X_k = x_k)]$ and $\hat{r}(x_k, a)$ is the estimator of $\mathbf{E}[R|X=x, A=a]$.


The following analysis is done under certain causal assumptions:
\begin{enumerate}
    \item Modified positivity requirement: If for any $x_k \in \mathcal{S}$, $\pi(a|x_k) > 0$, then $\pi_z(a|x_k) > 0$ 
    %ii) consistency: If we assign context $k$ action $a_k$, then the reward $r_k$ observed is the reward for context $k$ under $a_k$ (i.e. we assume no interference between contexts);
    \item  Exchangeability: The measured covariates completely determine the context's probability of a given action (i.e. no unmeasured factors influence the assignment mechanism)
\end{enumerate}


\begin{theorem}
For any $x_k \in \mathcal{S}$, if $\mathbf{E}[\widehat{r}(x_k,a)] = \mathbf{E}[r(x_k,a)]$ or $\mathbf{E}[\hat{\pi}_{z}(a|x_k)] = \pi_{z}(a|x_k)$, then  $$\mathbf{E}_{(X, A, R) \sim D}[\widehat{V}(\pi, \mathcal{S})] = V(\pi, \mathcal{S}).$$
\end{theorem}

\begin{proof}
For any $x_k \in \mathcal{S}$, if $\mathbf{E}[\widehat{r}(x_k,a)] = \mathbf{E}[r(x_k,a)]$, then:

\begin{eqnarray*}
\mathbf{E}_{(X, A, R) \sim D}[\widehat{V}(\pi, \mathcal{S})] &=&    \mathbf{E}_{X_\mathcal{S}}\Bigg[\sum_{a \in \mathcal{A}} \pi(a|x_k)\mathbf{E}[\hat{r}(x_k, a)] + \sum_{a \in \mathcal{A}} \mathbf{E}\bigg[ 
\mathbb{I}_{\{A_k = a\}}\frac{\pi(a|x_k)}{\hat{\pi}_{z}(a|x_k)} \\
&\times&  (r(x_k, a) - \mathbf{E}[\hat{r}(x_k, a)]) \bigg] \Bigg] = \mathbf{E}_{X_\mathcal{S}}\left[\sum_{a \in A} \pi(a|x_k) r(x_k, a)\right]\\
&=& V(\pi, \mathcal{S}).
\end{eqnarray*}
On the other hand, if $\mathbf{E}[\hat{\pi}_{z}(a|x_k)] = \pi_{z}(a|x_k)$, then:
\begin{eqnarray*}
\mathbf{E}_{(X, A, R) \sim D}[\widehat{V}(\pi, \mathcal{S})] &=&  \mathbf{E}_{X_\mathcal{S}}\Bigg[\sum_{a \in A} \pi(a|x_k) \mathbf{E}[\hat{r}(x_k, a)] + \mathbf{E} \bigg[  (r(x_k, a) - \hat{r}(x_k, a))   \\
& & \sum_{a \in A} \frac{\pi(a|x_k)}{\hat{\pi}_{z}(a|x_k)} \mathbf{E} \left[ \mathbb{I}_{\{A_k =a\}}\right]\bigg] \Bigg]\\
&=& \mathbf{E}_{X_\mathcal{S}}\left[ \sum_{a \in A} \pi(a|x_k) \mathbf{E}[\hat{r}(x_k, a)] + \sum_{a \in A}   \pi(a|x_k) \mathbf{E} \left[ r(x_k, a) - \hat{r}(x_k, a)\right] \right]\\
&=& \mathbf{E}_{X_\mathcal{S}}\left[\sum_{a \in A}   \pi(a|x_k) r(x_k, a)  \right]\\
&=& V(\pi, \mathcal{S}).
\end{eqnarray*}
\end{proof}

\begin{corollary}
In a parametric setting, $\widehat{V}(\pi, \mathcal{S})$ is consistent if the parametric form is correctly specified for either $r(x_k,a)$ or $ \pi_{z}(a|x_k)$ (i.e. one of the nuisance functions can be mispecified).
\end{corollary}


\begin{definition}
Let $\mathbf{P}[\widehat{f}(X)]$ denote $\mathbf{E}[\widehat{f}(X) |D^n]$ where  $D^n$ is the sample used to construct the random function $\hat{f}$. %Then the only randomness in $\mathbf{P}[\widehat{f}(X)]$ is over the input X.
\end{definition}

\begin{definition}
Define $\mathbb{L}_2(\mathbf{P})$ norm to be $||\hat{f} ||_{\mathbf{P}}=\sqrt{\mathbf{P}(\hat{f}^2)}$.
\end{definition}

\begin{theorem}
For all $x_k \in \mathcal{S}$ and $a \in \mathcal{A}$, if $||\mathbf{E}[r(x_k, a)] - \hat{r}(x_k, a)||_{\mathbf{P}} = o_p(n^{-\frac{1}{4}})$,
\\ $||\pi_z(x_k, a) - \hat{\pi}_z(x_k, a)||_{\mathbf{P}} = o_p(n^{-\frac{1}{4}})$ and $\frac{\pi(a|x_k)}{\hat{\pi}_z(a|x_k)} <  C$ for some constant C, then $\widehat{V}(\pi, \mathcal{S})$ is $\sqrt{n}$ consistent and asymptotically normal. Note that from our positivity requirement (causal assumption 1), the requirement that $\frac{\pi(a|x_k)}{\hat{\pi}_z(a|x_k)}$ be bounded is mild.
\end{theorem}


\begin{proof}
We first define the following two quantities for all $x_k \in \mathcal{S}$:
$$\phi(\pi, x_k) = \sum_{a \in \mathcal{A}} \pi(a|x_k) r(x_k, a)$$ where $\phi(\pi, x_k)$ does not depend on the sample histories, and  $$\widehat{\phi}(\pi, x_k) = \mathbb{I}{\{x_k \in \mathcal{S} \}} \times \left(\sum_{a \in \mathcal{A}} \pi(a|x_k)\hat{r}(x_k, a) + \sum_{a \in \mathcal{A}} \mathbb{I}_{\{A_k = a\}}\frac{\pi(a|x_k)}{\hat{\pi}_{z}(a|x_k)} \cdot (r(x_k, a) - \hat{r}(x_k, a)) \right).$$

Note that $V(\pi, \mathcal{S}) = \mathbf{P}_{X_\mathcal{S}} \left[ \phi(\pi, x_k) \right]$ and 
$\widehat{V}(\pi, \mathcal{S})  = \frac{1}{\sum_{k=1}^N \mathbb{I}{\{x_k \in \mathcal{S} \}}}  \sum_{k=1}^{N } \mathbb{I}{\{x_k \in \mathcal{S} \}} \widehat{\phi}(\pi, x_k)$.

We want to show $\widehat{V}(\pi, \mathcal{S}) - V(\pi, \mathcal{S}) = O_p(\frac{1}{\sqrt{n}})$. We have
\begin{align}
\widehat{V}(\pi, \mathcal{S}) - V(\pi, \mathcal{S}) &= \frac{1}{\sum_{k=1}^N \mathbb{I}{\{x_k \in \mathcal{S} \}}}  \sum_{k=1}^{N } \mathbb{I}{\{x_k \in \mathcal{S} \}} \widehat{\phi}(\pi, x_k) - \mathbf{P}_{X_\mathcal{S}} \left[ \phi(\pi, x_k) \right] \nonumber \\
&= \frac{1}{\sum_{k=1}^N \mathbb{I}{\{x_k \in \mathcal{S} \}}}  \sum_{k=1}^{N } \mathbb{I}{\{x_k \in \mathcal{S} \}} \widehat{\phi} (\pi, x_k) - \mathbf{P}_{X_\mathcal{S}} \left[ \phi(\pi, x_k) \right]  \nonumber \\
&+ \mathbf{P}_{X_\mathcal{S}} \left[ \widehat{\phi}(\pi, x_k) \right] - \mathbf{P}_{X_\mathcal{S}} \left[ \widehat{\phi}(\pi, x_k) \right] \nonumber \\
&+ \mathbf{P}_{X_\mathcal{S}} \left[ \phi(\pi, x_k) \right] - \frac{1}{\sum_{k=1}^N \mathbb{I}{\{x_k \in \mathcal{S} \}}}  \sum_{k=1}^{N } \mathbb{I}{\{x_k \in \mathcal{S} \}} \phi(\pi, x_k)   \nonumber \\
&+\frac{1}{\sum_{k=1}^N \mathbb{I}{\{x_k \in \mathcal{S} \}}}  \sum_{k=1}^{N } \mathbb{I}{\{x_k \in \mathcal{S} \}} \phi(\pi, x_k)  - \mathbf{P}_{X_\mathcal{S}} \left[ \phi(\pi, x_k) \right] \nonumber \\
&= \frac{1}{\sum_{k=1}^N \mathbb{I}{\{x_k \in \mathcal{S} \}}}  \sum_{k=1}^{N } \mathbb{I}{\{x_k \in \mathcal{S} \}} \phi(\pi, x_k)  - \mathbf{P}_{X_\mathcal{S}} \left[ \phi(\pi, x_k) \right] \label{eq1}\\
&+ \mathbf{P}_{X_\mathcal{S}} \left[ \widehat{\phi}(\pi, x_k) \right] - \mathbf{P}_{X_\mathcal{S}} \left[ \phi(\pi, x_k) \right]  \label{eq2}\\
&+ \mathbf{P}_{X_\mathcal{S}} \left[ \phi(\pi, x_k) - \widehat{\phi}(\pi, x_k) \right] - \frac{1}{\sum_{k=1}^N \mathbb{I}{\{x_k \in \mathcal{S} \}}}  \nonumber  \\
&\times \sum_{k=1}^{N }\mathbb{I}{\{x_k \in \mathcal{S} \}} \left(\phi(\pi, x_k) - \widehat{\phi} (\pi, x_k) \right)\label{eq3}
\end{align}

\eqref{eq1} can be directly bounded using Central Limit Theorem, which suggests that \eqref{eq1} is asymptotically normal and  $O_p(\frac{1}{\sqrt{n}})$. \eqref{eq2} will be analyzed in lemma \ref{core-lemma}: under the condition that the nuisance functions are of $o_p(n^{-\frac{1}{4}})$ with respect to their true value, \eqref{eq2} is $o_p(\frac{1}{\sqrt{n}})$. \eqref{eq3} are $o_p(n^{-\frac{1}{4}})$ under smoothness assumptions on $\phi$ \cite{van2000asymptotic} or under sample-splitting \cite{kennedy2018sharp}: see lemmas \ref{vandervaart} and \ref{kennedy} below. 
\end{proof}



\begin{lemma} \label{core-lemma}
For all $x_k \in \mathcal{S}$ and $a \in \mathcal{A}$, if $||\mathbf{E}[r(x_k, a)] - \hat{r}(x_k, a)||_{\mathbf{P}} = o_p(n^{-\frac{1}{4}})$,
\\ $||\pi_z(x_k, a) - \hat{\pi}_z(x_k, a)||_{\mathbf{P}} = o_p(n^{-\frac{1}{4}})$ and $\frac{\pi(a|x_k)}{\hat{\pi}_z(a|x_k)} <  C$ for some constant C, then
$$\mathbf{P}[\widehat{V}(\pi, \mathcal{S})] - V(\pi, \mathcal{S}) = o_p(\frac{1}{\sqrt{n}}).$$
\end{lemma}

\begin{proof}
%$$\mathbb{P}\left( \mathbf{E}_{(x_k, a_k, r_k) \sim D}[\widehat{g}(\pi, \mathcal{V}) (x)] - g(\pi,\mathcal{V})(x) \right) \to 0.$$
\begin{eqnarray*}
\mathbf{P}[\widehat{V}(\pi, \mathcal{S})]  &-& V(\pi, \mathcal{S})\\ 
 &=& \mathbf{P}\left[\sum_{a \in A} \pi(a|x_k) \hat{r}(x_k, a) +   (r(x_k, a) - \hat{r}(x_k, a))   \sum_{a \in A} \frac{\pi(a|x_k)}{\hat{\pi}_{z}(a|x_k)}  \mathbb{I}_{\{A_k =a\}} \right] \\
 &-& \mathbf{P} \left[ \sum_{a \in \mathcal{A}} \pi(a|x_k) r(x_k, a)  \right] \\
 &=& \mathbf{P} \left[  \sum_{a \in \mathcal{A}} \pi(a|x_k) \left[ \hat{r}(x_k, a) - \mathbf{E}[r(x_k,a)] +    \frac{\pi_z(a|x_k)}{\hat{\pi}_{z}(a|x_k)} (\mathbf{E}[r(x_k, a)] - \hat{r}(x_k, a))  \right] \right]\\
 &=& \mathbf{P} \left[  \sum_{a \in \mathcal{A}} \pi(a|x_k)    \frac{(\mathbf{E}[r(x_k, a)] - \hat{r}(x_k, a)) (\pi_z(a| x_k) - \hat{\pi}_z(a| x_k))}{\hat{\pi}_z(a| x_k)}
 \right]\\
 &=& o_p(\frac{1}{\sqrt{n}})
\end{eqnarray*}


where the second equality follows from iterated expectation and the final equality follows from our conditions that $\frac{\pi(a|x_k)}{\hat{\pi}_z(a|x_k)} <  C$ for some constant C, and
$||\mathbf{E}[r(x_k, a)] - \hat{r}(x_k, a)||_{\mathbf{P}} = o_p(n^{-\frac{1}{4}})$,
\\ $||\pi_z(x_k, a) - \hat{\pi}_z(x_k, a)||_{\mathbf{P}}= o_p(n^{-\frac{1}{4}})$ for all $x_k \in \mathcal{S}$ and $a \in \mathcal{A}$.

\end{proof}

We note we can satisfy these conditions on the nuisance functions using nonparametric techniques if for example, we have a Holder class of functions $H(\beta, L)$ where $\beta \geq d/2$ since the Holder minimax rate is $n^{\frac{-\beta}{d+2\beta}}$; we can alternatively satisfy these conditions with sparsity assumptions on the nuisance functions (i.e. that there is an s-sparse basis expansion) or if we have a class of functions with bounded total variation.

\begin{lemma} \label{vandervaart}
(Van der Vaart) If the function class of $\phi$ is Donsker (e.g. Sobolev, Holder) then $\mathbf{P}_{X_\mathcal{S}} \left[ \phi(\pi, x_k) - \widehat{\phi}(\pi, x_k) \right] - \frac{1}{\sum_{k=1}^N \mathbb{I}{\{x_k \in \mathcal{S} \}}}  
\times \sum_{k=1}^{N }\mathbb{I}{\{x_k \in \mathcal{S} \}} \left(\phi(\pi, x_k) - \widehat{\phi} (\pi, x_k) \right) = o_p(\frac{1}{\sqrt{n}})$
\end{lemma}
\begin{proof}
Proof outline: By the Donsker assumption, the empirical process converges in probability to a Gaussian process. By continuous mapping theorem, then we conclude the empirical process is $o_p(\frac{1}{\sqrt{n}})$. See Lemma 19.24 in \cite{van2000asymptotic} for more details.
\end{proof}

\begin{lemma} \label{kennedy}
(Kennedy) Let the nuisance functions in $\hat{\phi}$ be estimated using a sample $D^n= z_1,.... z_k$ independent from the empirical measure which is taken over $z_{k+1}, ... z_N$. Then, if $|| \hat{\phi} - \phi||_{\mathbf{P}}$, we have that  $\mathbf{P}_{X_\mathcal{S}} \left[ \phi(\pi, x_k) - \widehat{\phi}(\pi, x_k) \right] - \frac{1}{\sum_{k=1}^N \mathbb{I}{\{x_k \in \mathcal{S} \}}}  
\times \sum_{k=1}^{N }\mathbb{I}{\{x_k \in \mathcal{S} \}} \left(\phi(\pi, x_k) - \widehat{\phi} (\pi, x_k) \right) = o_p(\frac{1}{\sqrt{n}})$. See \cite{kennedy2018sharp} for proof details.
\end{lemma}


%Bias result: If (for any $x_k$ in history since we have the positivity assumption) $\mathbf{E}[\widehat{r}(x,a)] = r(x,a)$ or $\mathbf{E}[\hat{\pi}_{z}(a|x)] = \pi_{z}(a|x)$, then  $$\mathbf{E}_{(x_k, a_k, r_k) \sim D}[\widehat{g}(\pi, \mathcal{V})] = g(\pi, \mathcal{V}).$$

%Consistency result: $$\mathbf{E}_{(x_k, a_k, r_k) \sim D}[\widehat{g}(\pi, \mathcal{V})] \overset{p}{\to} g(\pi,\mathcal{V}).$$







%Or are actually proving (which is more close to the sum of three items format):$$ \widehat{\mathbf{E}}_{(x_k, a_k, r_k) \sim D}[\widehat{g}(\pi, \mathcal{V}) (x)]  \overset{p}{\to} g(\pi,\mathcal{V})(x)?$$

%So, here, there's something off... what exactly should the expectation been taken for? I guess my problem is somehow related to how we deal with $\mathcal{V}.$ Maybe it's easier to just say $x_k$ instead of $\mathcal{V}$?

%For any $x$, $$\widehat{g}(x) - g(x) = \widehat{g}(x) - \mathbf{E}[\widehat{g}(x)] + \mathbf{E}[\widehat{g}(x)] - g(x)$$ or are we proving : $$\widehat{\mathbf{E}}[\widehat{g}(x)] - \mathbf{E}[g(x)] = ?$$ But is the second term?
\section{Finding Maximally Advantaged and Disadvantaged Subgroups} \label{finding}
For certain applications, we may have a pre-defined subgroup of interest, such as a vulnerable or protected population. In other contexts, we may want to know which subgroup out of all possible subgroups benefit most and which suffer most under a proposed policy. We want to partition the feature space such that we maximize the variance in the $\mathcal{S}$-specific effects of the resulting subgroups. This is exactly what a regression tree using the estimated conditional value function as the outcome  does (in a local, greedy manner) since minimizing mean squared error on the split is equivalent to maximizing variance in the estimates from the split \cite{wager2017estimation}. Let $\widehat{V}(X_i)$ be the estimate of value function for the training point $X_i$ using $\widehat{V}_{DR}(\pi, \mathcal{S})$ (defined in section \ref{drest}). Let $\hat{\mu}(X_i)$ be the prediction of the tree. Since 

\begin{equation*}
\sum_{i=1}^N \hat{\mu}(X_i) = \sum_{i=1}^N \widehat{V}(X_i)
\end{equation*}

Then we have that the mean squared error (MSE) is 
\begin{eqnarray*}
&=&\frac{1}{N} \sum_{i=1}^N (\hat{\mu}(X_i) - \widehat{V}(X_i))^2 
\\&=& \frac{1}{N}\sum_{i=1}^N \widehat{V}(X_i)^2 - \frac{1}{N} \sum_{i=1}^N \hat{\mu}(X_i)^2 
\\&=&\frac{1}{N}\sum_{i=1}^N \widehat{V}(X_i)^2 - \text{Var}(\hat{\mu}(X_i)) - (1/N \sum_{i=1}^N \widehat{V}(X_i))^2
\end{eqnarray*}

Since  $\widehat{V}(X_i)$ for $i=1,...N$ are our fixed labels, we see that minimizing the MSE is equivalent to maximizing $\text{Var}(\hat{\mu}(X_i))$ over $i=1,...N$. This greedy splitting gives us a hierarchical and interpretable way to find subgroups that are most and least disadvantaged under the policy: if we want coarse subgroups, we look to the top of the tree; for granular subgroups, we look to the bottom of the tree. Following the feature splits gives us the exact definition of our subgroup of interest and the predicted value at that node gives us the $\mathcal{S}$-specific value function for that subgroup. Future work could investigate graph and set-partitioning methods that would provably find the most disadvantaged and advantaged subgroups.
\section{Evaluation of Nonstationary Policies} \label{nonstationary}
We can use our doubly robust estimator to estimate $\mathcal{S}$-specific value functions  on nonstationary (i.e. online) algorithms by adapting the rejection sampling approach of \cite{dudik2014doubly}. A nonstationary policy updates the policy based on the history. Formally, the nonstationary policy is $\pi(a_k |x_k, h_{k-1})$ where $h_{k-1} = (x_1, a_1, r_1), ...(x_{k-1}, a_{k-1}, r_{k-1})$ denotes the history up to $k-1$. Note that this target history is under the target policy, so in our offline evaluation setup, we do not observe this. However, we can simulate the target history using rejection sampling. As in \cite{dudik2014doubly} we assume perfect logging i.e. that our exploration policy is known. The exploration policy can be nonstationary or stationary; for generality we will use $\pi_{z,k}$ to denote the exploration policy at the time of observation of context $k$. This assumption is valid in a trial setting or for instance in applications where software engineers know the policy that shows news articles/ads/etc. Our goal is to estimate the value function on a subset $\mathcal{S}$ after T rounds of target history: 
\begin{equation}
    V_{1:N}(\pi, \mathcal{S}) = \mathbf{E}[\sum_{k=1}^N R_k]
\end{equation}
Assume we have set $\mathbf{S} = \{\mathcal{S}_1, ... \mathcal{S}_a \} $ of the $a$ subgroups for which we want to compute $\mathcal{S}_a$-specific effect. We have to run the simulation procedure only once, regardless of the size of $\mathbf{S}$. The algorithm proceeds exactly as Algorithm 1 of  \cite{dudik2014doubly} except that we modify steps 1-2:

For our observed history $k=1,2...N$ 
\begin{itemize}
    \item for each $\mathcal{S}_a \in \mathbf{S}$,
    \begin{itemize}
        \item if $x_k \in \mathcal{S}_a$,
        \begin{itemize}
            \item Update $\hat{V}(\mathcal{S}_a) \leftarrow \hat{V}(\mathcal{S}_a) + c_t \hat{V}_k$, \\
            where $\hat{V}_k = \hat{r}(x_k, \pi_t) + \frac{\pi_t(a_k|x_k)}{\pi_{z,k}(a_k |x_k)} (r_k - \hat{r}(x_k, a_k))$ and $c_t$ is a multiplier of the acceptance rate as updated in 6d of Algorithm 1 in \cite{dudik2014doubly}.
        \end{itemize}
    \end{itemize}
    \item Do steps 3-6 of Algorithm 1 in \cite{dudik2014doubly}.
\end{itemize}

\section{Conclusion}
The main contribution of this paper is a doubly robust estimator of $\mathcal{S}$-specific value functions. Doubly robust offline estimation of $\mathcal{S}$-specific value function can help us better understand how and where a policy will perform before investing time and money in online evaluation. These methods are also essential in assessing the fairness of policies.  
\bibliographystyle{plain}
\bibliography{citations}
\end{document}
